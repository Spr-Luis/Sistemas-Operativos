\documentclass[11pt,a4paper]{article}
\usepackage[utf8]{inputenc}
\usepackage[spanish]{babel}
\usepackage[left=3cm,right=3cm,top=4cm,bottom=3cm]{geometry}
\author{Chávez Soto Luis Armando}
\title{TAREA NO. 2}
\date{7 de Febrero del 2015}
\begin{document}

	\maketitle

\begin{abstract}
Trabajo de Investigación:

	\begin{itemize}
		\item ¿Qué es \textit{bootstrapping} a nivel de \textit{software} y \textit{hardware}?.
		\item ¿Qué es la arquitectura \textit{little-endian} y\textit{ big-endian}?.
		\item El procesador de tu equipo personal, ¿qué arquitectura utiliza?.	
	\end{itemize}
	 Realizar en LaTeX. 
\end{abstract}


\textbf{¿Qué es \textit{bootstrapping} a nivel de \textit{software} y \textit{hardware}?}
 
 $ $
 
Para definir que es el \textit{bootstrapping}, debemos establecer primeramente que el termino es utilizado en la estadística y en los negocios, pero para nuestro fines debemos establecer que será utilizado en la informática. Y esté es utilizado para describir el proceso de inicio de cualquier computador. Es referido al programa que inicializa el sistema operativo.

El término originalmente se utilizó a principios de los \textit{50's}, y era referido al botón de arranque para cargar un programa más pequeño que a su vez cargaría uno más amplio como un sistema operativo. El término se deriva de la expresión \begin{center}\textit{ "Pulling yourself up by your own bootstraps"}\end{center}

A nivel del hardware podríamos decir que refiere al ser ejecutado inmediatamente que termina el proceso POST (Power On Self Test), en donde se verifica e inicializa todos los componentes de entrada y salida de cómputo así como diagnosticar el estado del hardware y si esté carga satisfactoriamente el bootstrapping hace su aparición.

\begin{itemize}
\item GNU grand unified bootloader (GRUB): Un multiboot que especifica y permite al usuario escoger el sistema operativo. 
\item NT loader (NTLDR): Un bootloader de Microsoft Windows NT OS que usualmente corre desde el disco duro.
\item Linux loader (LILO): Un bootloader de Linux que por lo general corre desde un disco duro o una unidad de cd. 
\end{itemize}

A nivel de software el \textit{bootstrapping} puede referirse a la forma en como se preparan los entornos de programación. En donde la cuestión es preparar programas complejos con un editor de texto simple e ir creando lenguajes más sofisticados y complejos.

$ $

\textbf{¿Qué es la arquitectura \textit{little-endian} y\textit{ big-endian}?}
 
 $ $

Básicamente se define como se representa la información, la forma en la que se almacena en el computador y esto puede cobrar una especial relevancia cuando se trata de trabajar con datos de más de un byte.

La representación de los datos en bytes de define mediante \textit{el byte más representativo(MSB)} o \textit{el menos representativo (LSB)} estos formatos son nombrados como \textbf{Big Endian} aquel que ordena los bytes del más significativo al menos significativo, y el \textbf{Little Endian} como el menos significativo al más significativo. 

$ $

\emph{Por ejemplo:}
$ $
La representación del siguiente número hexadecimal  \textbf{\textit{0x74726563650d0a}}, sería de la forma:
$ $
\begin{center}
0x74 0x72 0x65 0x63 0x65 0x0d 0x0a - Big Endian


0x0a 0x0d 0x65 0x63 0x65 0x72 0x74 - Little Endian
\end{center}

$ $

\textbf{El procesador de tu equipo personal, ¿qué arquitectura utiliza?}
 
 $ $
 
 El procesador de mi computador es un Intel Core i5 a 2.6 GHz, bajo un sistema UNIX en donde utiliza una arquitectura Big Endian.
 
 $ $
 
 $ $

 $ $

 $ $
 
 $ $
 
 $ $

 $ $

 $ $
\textbf{Bibliografía}
 
\begin{itemize}
		\item Bootstrap, Technopedia - http://www.techopedia.com/definition/3328/bootstrap.
		\item Los formatos Big Endian y Little Endian - http://www.arumeinformatica.es/blog/los-formatos-big-endian-y-little-endian/
		\item Bootstrapping, Wikipedia - http://es.wikipedia.org/wiki/Bootstrapping
	\end{itemize} 
 
\end{document}
