\documentclass[11pt,a4paper]{article}
\usepackage[utf8]{inputenc}
\usepackage[spanish]{babel}
\usepackage[left=4cm,right=3cm,top=4cm,bottom=3cm]{geometry}
\author{Chávez Soto Luis Armando}
\title{TAREA NO. 2}
\date{7 de Febrero del 2015}
\begin{document}

	\maketitle

\begin{abstract}
Trabajo de Investigación:

	\begin{itemize}
		\item ¿Qué es \textit{bootstrapping} a nivel de \textit{software} y \textit{hardware}?.
		\item ¿Qué es la arquitectura \textit{little-endian} y\textit{ big-endian?}.
		\item El procesador de tu equipo personal, ¿qué arquitectura utiliza?.	
	\end{itemize}
	 Realizar en LaTeX. 
\end{abstract}


\textbf{¿Qué es \textit{bootstrapping} a nivel de \textit{software} y \textit{hardware}?.}
 
 
Para definir que es el bootstrapping, debemos establecer primeramente que el termino es utilizado en la estadística y en los negocios, pero para nuestro fines debemos establecer que será utilizado en la informatica. Y esté es utilizado para descibir el proceso de incio de cualquier computador. Es referido al programa que inicializa el sistema operativo.

El término originalmente se utilzó a principios de los 50's, 


\end{document}
